\documentclass{article}
\usepackage[english]{babel}
\usepackage[utf8]{inputenc}
\usepackage{amsmath}

%%%%%%%%%% Start TeXmacs macros
\newcommand{\tmaffiliation}[1]{\\ #1}
\newcommand{\tmop}[1]{\ensuremath{\operatorname{#1}}}
\newcommand{\tmtextbf}[1]{{\bfseries{#1}}}
%%%%%%%%%% End TeXmacs macros

\begin{document}

\title{Chapter 4 Confidence Interval Estimation}

\author{
  Yuejina Mo
  \tmaffiliation{May 21, 2018}
}

\maketitle

\section*{Part II}

\tmtextbf{Q4.8} A process produces bags of refined sugar. The weights of the
contents of these bags are normally distributed with standard deviation 1.2
ounces. The contents of a random sample of 25 bags had a mean weight of 19.8
ounces. Find the upper and lower confidence limits of a 99\% confidence
interval for the true mean weight for all bags of sugar produced by the
process.

Ans:

For a 99\% confidence interval the reliability factor is $Z_{0.005} = 2.58$
and with a sample mean of 19.8, n=25, and a standard deviation of 1.2, the
confidence limits are as follows:
\[ \tmop{Upper} \tmop{confidence} \tmop{limit} = \overset{-}{x} + Z_{\alpha /
   2} \frac{\sigma}{\sqrt{n}} = 19.8 + 2.58 \frac{1.2}{\sqrt{25}} = 20.42 \]
\[ \tmop{Lower} \tmop{confidence} \tmop{limit} = \overset{-}{x} - Z_{\alpha /
   2} \frac{\sigma}{\sqrt{n}} = 19.8 - 2.58 \frac{1.2}{\sqrt{25}} = 19.18 \]
Source: 國立交通大學, 唐麗英老師上課講義,
http://ocw.nctu.edu.tw/course/stat021/CH7.pdf

\end{document}
