\documentclass{article}
\usepackage[english]{babel}
\usepackage{amsmath,amssymb}

%%%%%%%%%% Start TeXmacs macros
\newcommand{\assign}{:=}
\newcommand{\nobracket}{}
\newcommand{\tmaffiliation}[1]{\\ #1}
\newcommand{\tmop}[1]{\ensuremath{\operatorname{#1}}}
%%%%%%%%%% End TeXmacs macros

\begin{document}

\title{Mathematical Statistics Assignment 2}

\author{
  yuejian mo
  \tmaffiliation{11510511}
}

\maketitle

\section*{Part II}

Q 2.9: Find the MLE of the unknown parameter $\theta$ when $X_1, X_2, \ldots,
X_n$ is a sample from the distribution whose density function is:
\[ f_X (x) = \frac{1}{2} e^{- | x - \theta |}, - \infty < x < \infty \]
Suggest Solution:
\[ f_X (x) = \frac{1}{2} e^{- | x - \theta |}, - \infty < x < \infty \]
is a special case of the Laplace distribution given as follows:
\[ f_X (x | \mu, \sigma \nobracket) = \frac{1}{\sqrt{2} \sigma} e^{-
   \frac{\sqrt{2} | x - \mu |}{\sigma}}, x \in \Re \]
for $\sigma = \sqrt{2}$ and $\mu \assign \theta$. To be more general, lets
consider the Laplace distribution with parameter $(\mu, \sigma) .$

Consider the likelihood function for $N$ data samples:
\[ L (\mu, \sigma ; x) = \prod_{t = 1}^N \frac{1}{\sqrt{2} \sigma} e^{-
   \frac{\sqrt{2} | x_t - \mu |}{\sigma}} = \left( \sqrt{2} \sigma \right)^{-
   N} e^{\frac{- \sqrt{2}}{\sigma} \sum_{t = 1}^N | x_t - \mu |} \]
Take the log likelihood function as $l (\mu, \sigma ; x) = \log (L (\mu,
\sigma ; x))$ and we get
\[ l (\mu, \sigma ; x) = - N \tmop{In} \left( \sqrt{2} \sigma \right) -
   \frac{\sqrt{2}}{\sigma} \sum_{t = 1}^N | x_t - \mu | \]
Take the derivate with respect to the parameter $\mu$
\[ \frac{\partial l}{\partial \mu} = - \frac{\sqrt{2}}{\sigma} \sum_{t = 1}^N
   \frac{\partial | x_t - \mu |}{\partial \mu} \]
which is equal to
\[ = \frac{\sqrt{2}}{\sigma} \sum_{t = 1}^N \tmop{sgn} (x_t - \mu) \]
using the identity
\[ \frac{\partial | x |}{\partial x} = \frac{\partial \sqrt{x^2}}{\partial x}
   = x (x^2)^{- 1 / 2} = \frac{x}{| x |} = \tmop{sgn} (x) \]
To maximize the likelihood function we need to solve
\begin{equation}
  = \frac{\sqrt{2}}{\sigma} \sum_{t = 1}^N \tmop{sgn} (x_t - \mu) = 0
\end{equation}
For which we have two cases: $N$ is even or odd.

If $N$ is odd and we choose $\hat{\mu} = \tmop{median} (x_1, \cdots x_N)$,
then there are $\frac{N - 1}{2}$ cases where $x_t < \mu$ and for the other
$\frac{N - 1}{2}$ cases $x_t > \mu$, therefore $\hat{\mu}$ satisfies (1) and
is the Maximum likelihood estimator for the parameter $\mu$.

If $N$ is even, we can not simply choose one $x_t$ which will satisfy(1),
however we can still minimize it through ranking the observations as $x_1
\leqslant x_2 \leqslant \ldots, x_N$ \ and then choosing either $x_{N / 2}$ or
$x_{(N + 1) / 2}$.

In summary $\hat{\mu} = \tmop{median} (x_1, \ldots, x_N)$ is the maximum
likelihood estimator for any $N$.

\section*{Sources}

https://math.stackexchange.com/questions/240496/finding-the-maximum-likelihood-estimator

\

\end{document}
